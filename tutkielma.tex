% --- Template for thesis / report with tktltiki2 class ---
% 
% last updated 2013/02/15 for tkltiki2 v1.02

\documentclass[finnish]{tktltiki2}

% --- General packages ---

\usepackage[utf8]{inputenc}
\usepackage[T1]{fontenc}
\usepackage{lmodern}
\usepackage{microtype}
%\usepackage[table,xcdraw]{xcolor}    % loads also »colortbl«
%\usepackage{listings}
\usepackage{minted}
\usepackage{amsfonts,amsmath,amssymb,amsthm,booktabs,enumitem,graphicx}
\usepackage{tocloft}
%\usepackage{relsize}
\usepackage[pdftex,hidelinks]{hyperref}
\usepackage[title]{appendix}
%\usepackage{tabularx}
%\usepackage[table]{xcolor}    % loads also »colortbl«
%\usepackage{float}

%\listfiles

\linespread{1.3}

\setlength{\intextsep}{18pt plus 2.0pt minus 2.0pt}

%\lstset{%
%  language=[LaTeX]TeX,
%  basicstyle=\ttfamily,
%  breaklines=true,
%  columns=fullflexible,
%}

%\setlength{\arrayrulewidth}{0.6pt}

% Automatically set the PDF metadata fields
\makeatletter
\AtBeginDocument{\hypersetup{pdftitle = {\@title}, pdfauthor = {\@author}}}
\makeatother

% --- Language-related settings ---
%
% these should be modified according to your language

% babelbib for non-english bibliography using bibtex
\usepackage[fixlanguage]{babelbib}
\selectbiblanguage{finnish}

% add bibliography to the table of contents
\usepackage[nottoc]{tocbibind}
% tocbibind renames the bibliography, use the following to change it back
\settocbibname{Lähteet}

\declarebtxcommands{finnish}{%
    \def\btxurldatecomment#1{ [#1]}%
}

%\addto\captionsfinnish{
%  \renewcommand{\lstlistingname}{Listaus}
%  \renewcommand{\lstlistlistingname}{Listaukset}
%}

% --- Theorem environment definitions ---
\newtheorem{lau}{Lause}
\newtheorem{lem}[lau]{Lemma}
\newtheorem{kor}[lau]{Korollaari}

\theoremstyle{definition}
\newtheorem{maar}[lau]{Määritelmä}
\newtheorem{ong}{Ongelma}
\newtheorem{alg}[lau]{Algoritmi}
\newtheorem{esim}[lau]{Esimerkki}

\theoremstyle{remark}
\newtheorem*{huom}{Huomautus}

% --- Custom hyphenations ---
\hyphenation{}

% --- tktltiki2 options ---
%
% The following commands define the information used to generate title and
% abstract pages. The following entries should be always specified:

\title{Metaohjelmointi Python-kielellä}
\author{Mikko Koho}
\date{\today}
\level{Seminaarityö}

\abstract{Tiivistelmä.
}

\keywords{Python, metaohjelmointi}
\classification{
}
% classification according to ACM Computing Classification System (http://www.acm.org/about/class/)
                  % This is probably mostly relevant for computer scientists

% If the automatic page number counting is not working as desired in your case,
% uncomment the following to manually set the number of pages displayed in the abstract page:
%
%\numberofpagesinformation{59 sivua + 7 liitesivua}
%


\begin{document}
    
% --- Front matter ---

\frontmatter      % roman page numbering for front matter

\maketitle        % title page

\makeabstract     % abstract page

\tableofcontents  % table of contents

% --- Main matter ---

\mainmatter       % clear page, start arabic page numbering


%%%%%%%%%%%%%%%%%%%
\section{Johdanto}
%%%%%%%%%%%%%%%%%%%

Metaohjelmoinnilla tarkoitetaan klassisen määritelmän mukaan sellaisen tietokoneohjelman tekemistä, joka kirjoittaa uusia tietokoneohjelmia \cite{hazzard2013}. Tämä on kuitenkin melko yksinkertaistettu määritelmä, eikä metaohjelmointia ole helppo määritellä tarkasti. Toinen yleinen määritelmä esittää metaohjelmoinnin olevan ''tietokoneohjelma, joka manipuloi toisia ohjelmia ajon aikana'' \cite{hazzard2013}.

Tässä seminaarityössä tarkastellaan Python-ohjelmointikielen tarjoamia työkaluja metaohjelmointiin. Alussa käydään läpi Python-kielen perusteita ja tämän jälkeen tutustutaan metaohjelmointiin Python-kielellä. Metaohjelmoinnista tarkastellaan lähinnä suoritusaikaista metaohjelmointia.



%%%%%%%%%%%%%%%%%%%
\section{Python-ohjelmointikieli}
%%%%%%%%%%%%%%%%%%%

Ensimmäinen Python-kielen versio julkaistu 1991 [TODO: lähde]. Python-kielestä on nykyään käytössä eri versioita ja Python 2.7 on edelleen melko suosittu vaikka versio 3 on julkaistu jo 2008. Versio 3 ei ole yhteensopiva aiempien versioiden kanssa. Version 2:n suosion taustalla on se, että monet suositut kirjastot ja sovelluskehykset eivät ole siirtyneet versioon 3. Tämän tekstin esimerkit toimivat sekä Python 2.7:llä että Python 3:lla, ellei toisin mainta.

Pythonin suosio on kasvanut tasaisesti ja se on nykyään käytetyin kieli ohjelmoinnin perusteiden opetukseen Yhdysvaltojen yliopistoissa \cite{python-teaching}.

Python-ohjelmakoodia voidaan kääntää useilla eri kääntäjillä \cite{martelli2006python}. Käytetyin kääntäjä on CPython (Classic Python), joka kääntää alkuperäisen koodin Python-tavukoodiksi. Tavukoodia ajetaan C-kielellä toteutetulla virtuaalikoneella \cite{martelli2006python}. Standardikirjasto on toteutettu osittain C:llä ja osittain Pythonilla.

Muita suosittuja kääntäjiä ovat Java-tavukoodiksi kääntävä Jython sekä IronPython, joka kääntää Python-koodia .NET-ympäristön käyttämäksi CIL-tavukoodiksi. PyPy on Python-kielellä toteutettu useissa eri ympäristöissä toimiva suoraan konekielelle koodia kääntävä ajonaikainen (just-in-time) kääntäjä. PyPy on toteutettu RPython-kielellä, joka on Python-kielen osajoukko.

Tässä luvussa käydään läpi Python-kielen perusteet.

\subsection{Pythonin syntaksi}

Python ohjelma koostuu loogisista riveistä, jotka ovat yhden tai useamman ''fyysisen'' rivin mittaisia. \cite{martelli2006python}. Loogisten rivien päättämiseen ei käytetä mitään merkkiä. Rivien sisennyksen perusteella erotetaan ohjelmakoodin lohkot toisistaan. Suositeltu tapa sisentää on käyttää ensimmäisen tason sisentämiseen 4 välilyöntiä ja seuraavaan 8 ja niin edelleen \cite{pep8}. 

\subsection{Operaattorit}

\subsection{Muuttujat}

Python on dynaamisesti tyypitetty kieli eli muuttujien arvon tyyppiä ei tarvitse eksplisiittisesti määrittää vaan tyyppi määräytyy sen perusteella minkälainen arvo muuttujaan sijoitetaan. Muuttujan tyyppiä voi myös vaihtaa sijoittamalla siihen uuden eri tyyppisen arvon. Listaus \ref{lst:ex1} sisältää yksinkertaisen esimerkin Python-kielen syntaksista.

Modernit IDE:t kuten PyCharm\footnote{\url{https://www.jetbrains.com/pycharm/}} pystyvät koodin perusteella usein päättelemään muuttujan tyypin.

\begin{listing}
    \inputminted{python}{code/foo.py}
    \label{lst:ex1}
    \caption{Yksinkertainen esimerkki Python-kielen syntaksista.}
\end{listing}

Pythonissa kaikki arvot, muuttujat ja funktiot ovat olioita. Olion tyyppi määrittää mitä metodeja ja ominaisuuksia olio tarjoaa. Osa olioista on muuttumattomia (immutable) ja osa muutettavia (mutable).

\subsection{Luokat ja oliot}

%TODO: Luokat ja oliot

Python on olio-ohjelmointikieli, joka tukee moniperintää.

Kahdella alaviivalla nimen alussa ja lopussa merkitään Python-kielen ''maagisia'' metodeita, attribuutteja ja olioita.

%TODO: Esimerkkejä

\subsection{Muuttujatyyppejä}

\subsection{Iteroitavat}

Listakehitelmä (list comprehension), joukkokehitelmä (set comprehension) ja sanakirjakehitelmä (dictionary comprehension) ovat tapoja luoda lista-, joukko- tai sanakirjaolioita. 

Esimerkki listakehitelmän käytöstä sekä joistain Python-kielen funktioista on listauksessa \ref{lst:ex_listakehis}. Funktio \verb|range| palauttaa Python 2:ssa listan ja Python 3:ssa generaattoriolion, jota voidaan käyttää listan tapaan. Funktio \verb|all| tarkastaa kaikkien listan (tai muun iteroitavan olion) totuusarvon. Pythonissa kaikki oliot voidaan evaluoida totuusarvoina, jolloin lukujen tapauksessa aina luku ''0'' evaluoituu epätodeksi ja muut luvut todeksi.

% TODO: Selosta esimerkit kohta kohdalta...

\begin{listing}
    \inputminted{python}{code/luvut.py}
    \label{lst:ex_listakehis}
    \caption{Esimerkki funktion range käytöstä ja listakehitelmistä.}
\end{listing}



%%%%%%%%%%%%%%%%%%%
\section{Python-kielen metaohjelmointimaisia komponentteja}
%%%%%%%%%%%%%%%%%%%

\subsection{Introspektio}

Introspektiolla tarkoitetaan tietojen hakemista muistissa olevista olioista, moduuleista ja funktioista \cite{dive-into-python}.

Funktio \verb|type| palauttaa olion tietotyypin \cite{dive-into-python}. Tämä on yleensä luokka, mutta vanhan tyylisillä Python 2 -luokilla tämä on merkkijono ''instance''. Varsinainen luokka löytyy aina olion attribuutista \verb|__class__|, mutta Python 3:ssa \verb|type|:n käyttö on suositus.

Funktio \verb|dir| palauttaa listan olion attribuuteista, joihin kuuluu myös olion metodit. Funktio \verb|getattr| ottaa parametrina merkkijonon ja palauttaa parametrin nimisen attribuutin. Tällöin voidaan esimerkiksi kutsua metodeja oliosta, jonka rakennetta ei tunnetta vielä käännösvaiheessa \cite{dive-into-python}. Funktiolla \verb|isinstance| voidaan tarkistaa onko joku olio tietyn tyyppinen. 

Olioiden lyhyet kuvaukset (docstring) saa ajon aikana haettua niiden \verb|__doc__| -attribuutista. 

Olioita, luokkia ja funktioita voidaan muokata ajon aikana melko vapaasti. Esimerkki peruskirjaston \verb|dir| funktion ylikirjoittamisesta omalla funktiolla on listauksessa \ref{lst:ex_intro2}. Esimerkin oma funktio palauttaa sille annetut argumentit tekemättä niille mitään. Tämänkaltaisesta kirjastojen ja moduuleiden osien ajonaikaisesta muokkaamisesta käytetään myös tsesta käytetään myös termiä ''monkey patching''.

\begin{listing}
    \inputminted{python}{code/introspektio2.py}
    \label{lst:ex_intro2}
    \caption{Standardikirjaston funktion ylikirjoittaminen omalla funktiolla.}
\end{listing}

\subsection{Standardikirjaston inspect-moduuli}

% TODO: https://docs.python.org/2/library/inspect.html

\subsection{Metaluokat}

\subsection{Generaattorit}

Yield on metaohjelmointia.

\subsection{Lausekkeet}

% TODO: Kuorruttajat(?) (decorator) \cite{dubois2005nest}.

% TODO: Osittainsovellus (partial application).

%\subsection{Kontrollirakenteita}
%\subsection{Oliot}

%Funktionaaliset piirteet.



%%%%%%%%%%%%%%%%%%%
\section{Python-metaohjelmointi laajemmin}
%%%%%%%%%%%%%%%%%%%

% TODO: Exec. Eval. Compile.

Käännösaikaista metaohjelmointia ei ole suoraan tuettu Pythonissa, mutta tämä ominaisuus on lisätty ainakin kahteen Pythonista jatkokehitettyyn kehitettyyn kieleen, Mythoniin\cite{mython} ja Convergeen\cite{tratt05}. Mython kääntää ohjelmakoodia suoraan Python-tavukoodiksi, mutta Convergen tuottamaa koodia ajetaan kielen omalla virtuaalikoneella.


\subsection{Reflektio}

In computer science, reflection is the ability of a computer program to examine (see type introspection) and modify the structure and behavior (specifically the values, meta-data, properties and functions) of the program at runtime.


%%%%%%%%%%%%%%%%%%%%%
\section{Yhteenveto}
%%%%%%%%%%%%%%%%%%%%%

Yhteenveto.


\pagebreak

% --- References ---
%
% bibtex is used to generate the bibliography. The babplain style
% will generate numeric references (e.g. [1]) appropriate for theoretical
% computer science. If you need alphanumeric references (e.g [Tur90]), use
%
% \bibliographystyle{babalpha-lf}
%
% instead.

\bibliographystyle{babalpha-lf}
\bibliography{references-fi}

\lastpage

\end{document}
