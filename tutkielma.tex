% --- Template for thesis / report with tktltiki2 class ---
% 
% last updated 2013/02/15 for tkltiki2 v1.02

\documentclass[finnish]{tktltiki2}

% --- General packages ---

\usepackage[utf8]{inputenc}
\usepackage[T1]{fontenc}
\usepackage{lmodern}
\usepackage{microtype}
%\usepackage[table,xcdraw]{xcolor}    % loads also »colortbl«
\usepackage{amsfonts,amsmath,amssymb,amsthm,booktabs,enumitem,graphicx}
\usepackage{tocloft}
%\usepackage{relsize}
\usepackage[pdftex,hidelinks]{hyperref}
\usepackage[title]{appendix}
\usepackage{tabularx}
\usepackage[table]{xcolor}    % loads also »colortbl«
%\usepackage{float}
\usepackage{listings}

%\listfiles

\linespread{1.3}

\setlength{\intextsep}{18pt plus 2.0pt minus 2.0pt}

\lstset{%
  language=[LaTeX]TeX,
  basicstyle=\ttfamily,
  breaklines=true,
  columns=fullflexible,
}

%\setlength{\arrayrulewidth}{0.6pt}

% Automatically set the PDF metadata fields
\makeatletter
\AtBeginDocument{\hypersetup{pdftitle = {\@title}, pdfauthor = {\@author}}}
\makeatother

% --- Language-related settings ---
%
% these should be modified according to your language

% babelbib for non-english bibliography using bibtex
\usepackage[fixlanguage]{babelbib}
\selectbiblanguage{finnish}

% add bibliography to the table of contents
\usepackage[nottoc]{tocbibind}
% tocbibind renames the bibliography, use the following to change it back
\settocbibname{Lähteet}

\declarebtxcommands{finnish}{%
    \def\btxurldatecomment#1{ [#1]}%
}

% --- Theorem environment definitions ---
\newtheorem{lau}{Lause}
\newtheorem{lem}[lau]{Lemma}
\newtheorem{kor}[lau]{Korollaari}

\theoremstyle{definition}
\newtheorem{maar}[lau]{Määritelmä}
\newtheorem{ong}{Ongelma}
\newtheorem{alg}[lau]{Algoritmi}
\newtheorem{esim}[lau]{Esimerkki}

\theoremstyle{remark}
\newtheorem*{huom}{Huomautus}

% --- Custom hyphenations ---
\hyphenation{}

% --- tktltiki2 options ---
%
% The following commands define the information used to generate title and
% abstract pages. The following entries should be always specified:

\title{Metaohjelmointi Python-kielellä}
\author{Mikko Koho}
\date{\today}
\level{Seminaariteksti}
%\level{Pro gradu -tutkielmasuunnitelma}

\abstract{Tiivistelmä.
}

\keywords{Python, metaohjelmointi}
\classification{
}
% classification according to ACM Computing Classification System (http://www.acm.org/about/class/)
                  % This is probably mostly relevant for computer scientists

% If the automatic page number counting is not working as desired in your case,
% uncomment the following to manually set the number of pages displayed in the abstract page:
%
%\numberofpagesinformation{59 sivua + 7 liitesivua}
%


\begin{document}
    
% --- Front matter ---

\frontmatter      % roman page numbering for front matter

\maketitle        % title page

\makeabstract     % abstract page

\tableofcontents  % table of contents

% --- Main matter ---

\mainmatter       % clear page, start arabic page numbering


%%%%%%%%%%%%%%%%%%%
\section{Johdanto}
%%%%%%%%%%%%%%%%%%%

Johdanto.

%%%%%%%%%%%%%%%%%%%
\section{Python-kieli}
%%%%%%%%%%%%%%%%%%%

%\subsection{Historia}

Ensimmäinen Python-kielen versio julkaistu 1991 [TODO: lähde]. Python-kielestä on nykyään käytössä eri versioita. Python 2.7 on edelleen melko suosittu vaikka versio 3 on julkaistu jo 2008. Versio 3 ei ole yhteensopiva aiempien versioiden kanssa. Version 2:n suosion taustalla on se, että monet suositut kirjastot ja sovelluskehykset eivät ole siirtyneet versioon 3.

Pythonin suosio on kasvanut tasaisesti ja muun muassa se on nykyään käytetyin kieli ohjelmoinnin perusteiden opetukseen Yhdysvaltojen yliopistoissa \cite{python-teaching}.

Tässä luvussa käydään läpi Python-kielen perusteet.

\subsection{Yleistä}

Kielen yleiset piirteet \cite{martelli2006python}.

Ohjelmakoodin lohkot tunnistetaan sisennyksen perusteella.

Python on dynaamisesti tyypitetty. Modernit IDE:t kuten PyCharm\footnote{\url{https://www.jetbrains.com/pycharm/}} osaavat seurata koodin perusteella muuttujien tyyppejä ja yleensä pystyvät kertomaan minkä tyyppinen arvo muuttujassa kulloinkin on.

\subsection{Syntaksi}

\subsection{Operaattorit}

\subsection{Muuttujatyyppejä}

Kaikki muuttujat ovat olioita.


\subsection{Muuttujatyyppejä}

Metaluokat.


\subsection{Iteroitavat}

\section{Python-kielen metaohjelmointimaisia komponentteja}

\subsection{Generaattorit}

Listakehitelmä (list comprehension), joukkokehitelmä (set comprehension) ja sanakirjakehitelmä (dictionary comprehension) ovat

Yield on metaohjelmointia.

\subsection{Lausekkeet}

Kuorruttajat(?) (decorator) \cite{dubois2005nest}.

Osittainsovellus (partial application).

%\subsection{Kontrollirakenteita}
%\subsection{Oliot}

%Funktionaaliset piirteet.

\section{Reflektio}

\cite{dive-into-python}

\section{Python-metaohjelmointi laajemmin}

Psyco.

Exec. Eval. Compile.

Käännösaikainen metaohjelmointi. Löytyykö template-kieliä?

\subsection{Reflektio}

In computer science, reflection is the ability of a computer program to examine (see type introspection) and modify the structure and behavior (specifically the values, meta-data, properties and functions) of the program at runtime.

reflektio (dir(), \_\_luokat\_\_). 




\subsection{Ohjelman ajonaikainen muokkaus}

metodien korvaaminen toisilla osoittimia muuttamalla. Monkey patching.

%%%%%%%%%%%%%%%%%%%%%
\section{Yhteenveto}
%%%%%%%%%%%%%%%%%%%%%

Yhteenveto.


\pagebreak

% --- References ---
%
% bibtex is used to generate the bibliography. The babplain style
% will generate numeric references (e.g. [1]) appropriate for theoretical
% computer science. If you need alphanumeric references (e.g [Tur90]), use
%
% \bibliographystyle{babalpha-lf}
%
% instead.

\bibliographystyle{babalpha-lf}
\bibliography{references-fi}

\lastpage

\end{document}
