% --- Template for thesis / report with tktltiki2 class ---
% 
% last updated 2013/02/15 for tkltiki2 v1.02

\documentclass[finnish]{tktltiki2}

% --- General packages ---

\usepackage[utf8]{inputenc}
\usepackage[T1]{fontenc}
\usepackage{lmodern}
\usepackage{microtype}
%\usepackage[table,xcdraw]{xcolor}    % loads also »colortbl«
\usepackage{listings}
\usepackage{minted}
\usepackage{amsfonts,amsmath,amssymb,amsthm,booktabs,enumitem,graphicx}
\usepackage{tocloft}
%\usepackage{relsize}
\usepackage[pdftex,hidelinks]{hyperref}
\usepackage[title]{appendix}
%\usepackage{tabularx}
%\usepackage[table]{xcolor}    % loads also »colortbl«
%\usepackage{float}

%\listfiles

\linespread{1.3}

\setlength{\intextsep}{18pt plus 2.0pt minus 2.0pt}

%\lstset{%
%  language=[LaTeX]TeX,
%  basicstyle=\ttfamily,
%  breaklines=true,
%  columns=fullflexible,
%}

%\setlength{\arrayrulewidth}{0.6pt}

% Automatically set the PDF metadata fields
\makeatletter
\AtBeginDocument{\hypersetup{pdftitle = {\@title}, pdfauthor = {\@author}}}
\makeatother

% --- Language-related settings ---
%
% these should be modified according to your language

% babelbib for non-english bibliography using bibtex
\usepackage[fixlanguage]{babelbib}
\selectbiblanguage{finnish}

% add bibliography to the table of contents
\usepackage[nottoc]{tocbibind}
% tocbibind renames the bibliography, use the following to change it back
\settocbibname{Lähteet}

\declarebtxcommands{finnish}{%
    \def\btxurldatecomment#1{ [#1]}%
}

\renewcommand\listingscaption{Listaus}
\renewcommand\listoflistingscaption{Listaukset}

% --- Theorem environment definitions ---
\newtheorem{lau}{Lause}
\newtheorem{lem}[lau]{Lemma}
\newtheorem{kor}[lau]{Korollaari}

\theoremstyle{definition}
\newtheorem{maar}[lau]{Määritelmä}
\newtheorem{ong}{Ongelma}
\newtheorem{alg}[lau]{Algoritmi}
\newtheorem{esim}[lau]{Esimerkki}

\theoremstyle{remark}
\newtheorem*{huom}{Huomautus}

% --- Custom hyphenations ---
\hyphenation{}

% --- tktltiki2 options ---
%
% The following commands define the information used to generate title and
% abstract pages. The following entries should be always specified:

\title{Metaohjelmointi Python-kielellä}
\author{Mikko Koho}
\date{\today}
\level{Seminaarityö}

\abstract{Tiivistelmä.
}

\keywords{Python, metaohjelmointi}
\classification{
}
% classification according to ACM Computing Classification System (http://www.acm.org/about/class/)
                  % This is probably mostly relevant for computer scientists

% If the automatic page number counting is not working as desired in your case,
% uncomment the following to manually set the number of pages displayed in the abstract page:
%
%\numberofpagesinformation{59 sivua + 7 liitesivua}
%


\begin{document}
    
% --- Front matter ---

\frontmatter      % roman page numbering for front matter

\maketitle        % title page

\makeabstract     % abstract page

\tableofcontents  % table of contents

% --- Main matter ---

\mainmatter       % clear page, start arabic page numbering


%%%%%%%%%%%%%%%%%%%
\section{Johdanto}
%%%%%%%%%%%%%%%%%%%

Metaohjelmoinnilla tarkoitetaan klassisen määritelmän mukaan sellaisen tietokoneohjelman tekemistä, joka kirjoittaa uusia tietokoneohjelmia \cite{hazzard2013}. Tämä on kuitenkin melko yksinkertaistettu määritelmä, eikä metaohjelmointia ole helppo määritellä tarkasti. Toinen yleinen määritelmä esittää metaohjelmoinnin olevan ''tietokoneohjelma, joka manipuloi toisia ohjelmia ajon aikana'' \cite{hazzard2013}. 

Tärkeä osa metaohjelmointia on ohjelman ajonaikainen tilansa tarkastelu ja muuntelu eli reflektio. Reflektioon kuuluu käsite introspektio, jolla tarkoitetaan ajonaikaista muistissa olevien olioiden tarkastelua \cite{dive-into-python}.

Tässä seminaarityössä tarkastellaan Python-ohjelmointikielen tarjoamia työkaluja metaohjelmointiin. Alussa käydään läpi Python-kielen perusteita ja tämän jälkeen tutustutaan metaohjelmointiin Python-kielellä. Metaohjelmoinnista tarkastellaan lähinnä suoritusaikaista metaohjelmointia. 



%%%%%%%%%%%%%%%%%%%
\section{Python-ohjelmointikieli}
%%%%%%%%%%%%%%%%%%%

Ensimmäinen Python-kielen versio on julkaistu 1991. Python-kielestä on nykyään käytössä eri versioita ja Python 2.7 on edelleen melko suosittu vaikka versio 3 on julkaistu jo 2008. Versio 3 ei ole yhteensopiva aiempien versioiden kanssa. Version 2:n suosion taustalla on se, että monet suositut kirjastot ja sovelluskehykset eivät ole siirtyneet versioon 3. Tämän seminaarityön esimerkit toimivat sekä Python 2.7:llä että Python 3:lla, ellei toisin mainta. Esimerkeissä olevat rivin alusta alkavat kommentit sisältävät ohjelman tulosteita, koodia koskevat kommentit on laitettu kommentoitavan rivin perään.

Pythonin suosio on kasvanut tasaisesti ja se on nykyään käytetyin kieli ohjelmoinnin perusteiden opetukseen Yhdysvaltojen yliopistoissa \cite{python-teaching}.

Python-ohjelmakoodia voidaan kääntää useilla eri kääntäjillä \cite{martelli2006python}. Käytetyin kääntäjä on CPython (Classic Python), joka kääntää alkuperäisen koodin Python-tavukoodiksi. Muita suosittuja kääntäjiä ovat Java-tavukoodiksi kääntävä Jython sekä IronPython, joka kääntää Python-koodia .NET-ympäristön käyttämäksi CIL-tavukoodiksi. PyPy on Python-kielellä toteutettu useissa eri ympäristöissä toimiva suoraan konekielelle koodia kääntävä ajonaikainen (just-in-time) kääntäjä. PyPy on toteutettu RPython-kielellä, joka on Python-kielen osajoukko.

Pythonin standardikirjasto on toteutettu osittain C:llä ja osittain Pythonilla.

CPythonilla käännettyä tavukoodia voidaan ajaa C-kielellä toteutetulla virtuaalikoneella \cite{martelli2006python}. Virtuaalikone tulkitsee ajettaessa tavukoodia lause kerrallaan.

Tässä luvussa käydään läpi Python-kielen perusteita ja metaohjelmoinnin kannalta olennaisia asioita.


\subsection{Syntaksi}

Python ohjelma koostuu loogisista riveistä, jotka ovat yhden tai useamman ''fyysisen'' rivin mittaisia. \cite{martelli2006python}. Loogisten rivien päättämiseen ei käytetä mitään merkkiä. Rivien sisennyksen perusteella erotetaan ohjelmakoodin lohkot toisistaan. Suositeltu tapa sisentää on käyttää ensimmäisen tason sisentämiseen 4 välilyöntiä ja seuraavaan 8 ja niin edelleen \cite{pep8}. 

Pythonissa on 30 avainsanaa (keyword), jotka ovat kielen varattuja sanoja. Näitä ovat esimerkiksi funktio \verb|print| ja kielen rakenteissa käytetyt sanat kuten \verb|if|, \verb|and| ja \verb|class|. Pythonin standardikirjasto koostuu sisäänrakennettujen funktioiden lisäksi kokoelmasta eri tarkoituksiin soveltuvia moduuleita (module), jotka pitää tarvittaessa tuoda erikseen osaksi suoritettavaa ohjelmaa komennolla \verb|import|.

% TODO: operaattorit


\subsection{Muuttujat}

Python on dynaamisesti tyypitetty kieli eli muuttujien arvon tyyppiä ei tarvitse eksplisiittisesti määrittää vaan tyyppi määräytyy sen perusteella minkälainen arvo muuttujaan sijoitetaan. Muuttujan tyyppiä voi myös vaihtaa sijoittamalla siihen uuden eri tyyppisen arvon. Listaus \ref{lst:ex1} sisältää yksinkertaisen esimerkin Python-kielen syntaksista. Rivillä 1 asetetaan muuttujan \verb|a| arvoksi merkkijono \verb|Hello world!|, joka tulostetaan rivillä 2. Rivillä 5 tulostetaan merkkijonon \verb|a| pituus.

Modernit IDE:t kuten PyCharm\footnote{\url{https://www.jetbrains.com/pycharm/}} pystyvät koodin perusteella usein päättelemään muuttujan tyypin.

\begin{listing}
    \inputminted[linenos]{python}{code/foo.py}
    \caption{Yksinkertainen esimerkki Python-kielen syntaksista.}
    \label{lst:ex1}
\end{listing}

Pythonissa kaikki arvot, muuttujat ja funktiot ovat olioita. Olion tyyppi määrittää mitä metodeja ja ominaisuuksia olio tarjoaa. Osa olioista on muuttumattomia (immutable) ja osa muutettavia (mutable). Python-kielessä ei ole erikseen vakioita, mutta käytäntönä on käyttää muuttujan nimessä pelkästään isoja kirjaimia, jos muuttujan arvoa ei ole tarkoitus muuttaa.

Pythonin sisäänrakennettuja tietotyyppejä on mm. numeeriset \verb|int| ja \verb|float|, sekvenssityypit \verb|list|, \verb|str| ja \verb|tuple|, joukko \verb|set|, ''sanakirja'' \verb|dict| sekä tiedosto \verb|file|. Näiden lisäksi standardikirjaston moduuleista löytyy lisää tietotyyppejä kuten \verb|datetime| ja \verb|array|.

\subsection{Sekvenssit}

Sisäänrakennetuista tietotyypeistä mm. lista, merkkijono ja monikko (tuple) ovat sekvenssejä. Sekvenssit ovat iteroitavia (iterable) olioita eli ne kykenevät palauttamaan jäseniään yksi kerrallaan. Iteroitavia olioita ovat myös muut oliot, joissa on toteutettu jäseniä palauttava \verb|__iter__| -metodi. Iteroitavia olioita voidaan käyttää suoraan osana esimerkiksi \verb|for| -toistolauseissa. kuten esimerkiksi lauseessa \verb|for x in [1,2,3]: print(x)|.

Listakehitelmä (list comprehension), joukkokehitelmä (set comprehension) ja sanakirjakehitelmä (dictionary comprehension) ovat helppoja tapoja luoda lista-, joukko- tai sanakirjaolioita jonkin syötteen perusteella. Esimerkkejä listakehitelmän käytöstä sekä joidenkin Python-kielen funktioiden käytöstä on listauksessa \ref{lst:ex_listakehis}. Funktio \verb|range| palauttaa Python 2:ssa listan ja Python 3:ssa generaattoriolion, jota voidaan käyttää listan tapaan. Funktio \verb|all| tarkastaa kaikkien listan (tai muun iteroitavan olion) totuusarvon. Pythonissa kaikki oliot voidaan evaluoida totuusarvoina, jolloin lukujen tapauksessa aina luku ''0'' evaluoituu epätodeksi ja muut luvut todeksi.

% TODO: Selosta esimerkki
\begin{listing}
    \inputminted[linenos]{python}{code/luvut.py}
    \caption{Esimerkki funktion range käytöstä ja listakehitelmistä.}
    \label{lst:ex_listakehis}
\end{listing}


\subsection{Luokat ja oliot}

%TODO: Luokat ja oliot

Python on olio-ohjelmointikieli, joka tukee moniperintää.


% TODO: Vanhat luokat, uudet luokat

%TODO: Esimerkkejä



%%%%%%%%%%%%%%%%%%%
\section{Metaohjelmointi Pythonilla}
%%%%%%%%%%%%%%%%%%%

Metaohjelmointi on Pythonilla hyvin luontevaa, koska olioita voidaan yleensä muokata täysin vapaasti ajon aikana ja niistä saadaan paljon metatietoa ulos Pythonin peruskirjaston työkaluilla.

 
\subsection{Reflektio}

Funktio \verb|type| palauttaa olion tietotyypin \cite{dive-into-python}. Tämä on yleensä luokka, mutta vanhan tyylisillä Python 2 -luokilla tämä on merkkijono ''instance''. Varsinainen luokka löytyy aina olion attribuutista \verb|__class__|, mutta Python 3:ssa \verb|type|:n käyttö on suositus.

Funktio \verb|dir| palauttaa listan olion attribuuteista, joihin kuuluu myös olion metodit. Funktio \verb|getattr| ottaa parametrina merkkijonon ja palauttaa parametrin nimisen attribuutin. Tällöin voidaan esimerkiksi kutsua metodeja oliosta, jonka rakennetta ei tunnetta vielä käännösvaiheessa \cite{dive-into-python}. Funktiolla \verb|isinstance| voidaan tarkistaa onko joku olio tietyn tyyppinen. 

Olioiden lyhyet kuvaukset (docstring) saa ajon aikana haettua niiden \verb|__doc__| -attribuutista. Kahdella alaviivalla nimen alussa ja lopussa merkitään Python-kielen ''maagisia'' metodeita, attribuutteja ja olioita. 

Olioita, luokkia ja funktioita voidaan muokata ajon aikana melko vapaasti. Esimerkki peruskirjaston \verb|dir| funktion ylikirjoittamisesta omalla funktiolla on listauksessa \ref{lst:ex_intro2}. Esimerkin oma funktio palauttaa sille annetut argumentit tekemättä niille mitään. Tämänkaltaisesta kirjastojen ja moduuleiden osien ajonaikaisesta muokkaamisesta käytetään myös tsesta käytetään myös termiä ''monkey patching''.

% TODO: käytä .getattr('print')
\begin{listing}
    \inputminted[linenos]{python}{code/introspektio2.py}
    \caption{Standardikirjaston funktion ylikirjoittaminen omalla funktiolla.}
    \label{lst:ex_intro2}
\end{listing}


% TODO: https://web.archive.org/web/20120926041111/http://www.ibm.com/developerworks/library/l-pyint/index.html
\subsection{inspect-moduuli}

Pythonin standardikirjaston inspect-moduuli tarjoaa työkaluja olioiden tilan tutkimiseen ohjelman ajon aikana. 
% TODO: https://docs.python.org/2/library/inspect.html

\subsection{Dynaamiset luokat}

Pythonilla on mahdollista luoda uusia luokkia dynaamisesti ohjelman ajon aikana \verb|type|-funktiolla.

% TODO: http://python-3-patterns-idioms-test.readthedocs.org/en/latest/Metaprogramming.html

\subsection{Metaluokat}

%\subsection{Lausekkeet}

% TODO: Kuorruttajat(?) (decorator) \cite{dubois2005nest}.

% TODO: Osittainsovellus (partial application).

%\subsection{Kontrollirakenteita}
%\subsection{Oliot}

%Funktionaaliset piirteet.


%\subsection{Reflektio}

%In computer science, reflection is the ability of a computer program to examine (see type introspection) and modify the structure and behavior (specifically the values, meta-data, properties and functions) of the program at runtime.

% TODO: Exec. Eval. Compile.

Pythonissa on sisäänrakennetut funktiot \verb|compile|, \verb|eval| ja \verb|exec|, joiden avulla voidaan kääntää ja ajaa Python-koodia ohjelman ajon aikana \cite{codeobjects,martelli2006python}.

Esimerkki ajonaikaisesta koodin kääntämisestä on listauksessa \ref{lst:ex_compile}. Esimerkin rivillä 1 luodaan muuttuja, joka sisältää käännettävän koodin merkkijonona. Rivillä 2 käännetään koodi merkkijonosta tavukoodiksi käyttämällä funktiota \verb|compile|, jonka parametrit ovat käännettävä koodi, koodin sisältävän tiedoston nimi, jossa käytetään merkkijonoa \verb|'<string>'| tarkoittamaan, että koodia ei luettu tiedostosta ja käännöstila (mode), jossa \verb|'single'| kertoo, että käännetään yksi lause (statement). Rivi 4 tulostaa \verb|code_obj| -muuttujan merkkijonoesityksen. Rivillä 7 ajetaan tavukoodi muuttujasta \verb|code_obj|.

\begin{listing}
    \inputminted[linenos]{python}{code/dynamichello.py}
    \caption{Esimerkki Python-komennon kääntämisestä tavukoodiksi ohjelman ajon aikana ja käännetyn koodin ajamisesta \cite{codeobjects}.}
    \label{lst:ex_compile}
\end{listing}


\subsection{Tavukoodin tarkastelu ja muokkaus}

Standardikirjaston \verb|dis|-moduulilla voidaan tutkia Python-tavukoodia. Listauksessa \ref{lst:tavukoodi} eräs esimerkkifunktio sekä sen tavukoodin muodostaminen \verb|dis|-moduulin \verb|dis|-metodilla sekä tulostettu tavukoodi. Tavukoodi on tulostettu Python 2:lla ja se on erinäköinen Python 3:lla.

\begin{listing}
    \inputminted[linenos]{python}{code/tavukoodi.py}
    \caption{Python-tavukoodin tarkastelu dis-moduulilla.}
    \label{lst:tavukoodi}
\end{listing}

Moduuli \verb|parser| antaa rajapinnan Python-kääntäjän sisäiseen jäsennyspuuhun ja mahdollistaa sen muokkaamisen. Tämän moduulin lisäämisen jälkeen on kuitenkin standardikirjastoon lisätty \verb|ast|-moduuli, joka mahdollistaa kääntäjän abstraktin syntaksipuun (abstract syntax tree) luomisen annetun ohjelmakoodin perusteella ja sen muokkaamisen. Funktiolle \verb|compile| voidaan antaa parametrina muokatun AST-puun sisältävä \verb|AST|-olio ja kääntää se tavukoodiksi.

\subsection{Käännösaikainen metaohjelmointi}

Käännösaikaista metaohjelmointia ei ole suoraan tuettu Pythonin standardikirjastossa. Template-metaohjelmointi on mahdollista esimerkiksi Jinja2 -template-moottorilla (template engine) \cite{jinja}.

Käännösaikainen metaohjelmointi on myös lisätty suoraan osaksi kahta Pythonista jatkokehitettyä kieltä, Mythoniin\cite{mython} ja Convergeen\cite{tratt05}. Mython kääntää ohjelmakoodia suoraan Python-tavukoodiksi, mutta Convergen tuottamaa koodia ajetaan kielen omalla virtuaalikoneella.


%%%%%%%%%%%%%%%%%%%%%
\section{Yhteenveto}
%%%%%%%%%%%%%%%%%%%%%

Yhteenveto.


\pagebreak

% --- References ---
%
% bibtex is used to generate the bibliography. The babplain style
% will generate numeric references (e.g. [1]) appropriate for theoretical
% computer science. If you need alphanumeric references (e.g [Tur90]), use
%
% \bibliographystyle{babalpha-lf}
%
% instead.

\bibliographystyle{babalpha-lf}
\bibliography{references-fi}

\lastpage

\end{document}
