\documentclass{beamer}
\usepackage[T1]{fontenc}
\usepackage[utf8]{inputenc}
\usepackage[finnish]{babel}
\usepackage{listings}
\usepackage{minted}
\usepackage{verbatim}
\usetheme{Warsaw}

\title{Metaohjelmointi Python-kielellä}
\author{Mikko Koho}
\institute{Helsingin Yliopisto}
\date{\today}
%\beamertemplatenavigationsymbolsempty

\newcommand\blfootnote[1]{%
  \begingroup
  \renewcommand\thefootnote{}\footnote{#1}%
  \addtocounter{footnote}{-1}%
  \endgroup
}

\begin{document}

{
\setbeamertemplate{headline}{}
\begin{frame}
%  \titlepage
\maketitle
\end{frame}
}

%\section{Sisältö}

\begin{frame}
  \frametitle{Metaohjelmointi Python-kielellä}
  \tableofcontents%[currentsection]
\end{frame}

\section{Johdanto}

\begin{frame}[fragile]
\frametitle{Metaohjelmointi}
\begin{itemize}
%\item{Vaikeasti määriteltävä käsite}
\item{Tarkoittaa useita eri asioita}
\item{''Sellaisen ohjelman tekeminen, joka kirjoittaa uusia ohjelmia''}
\item{Ohjelma, joka tarkastelee tai manipuloi toisia ohjelmia tai itseään ajon aikana}
\item{Reflektio}
\item{Käännösaikainen metaohjelmointi}
\end{itemize}
\end{frame}

\begin{frame}[fragile]
\frametitle{Python}
\begin{itemize}
\item{Dynaaminen olio-ohjelmointikieli}
\item{Dynaamisesti tyypitetty}
\item{Ensimmäinen versio 1991}
\item{Nykyään käytössä versiot 2 ja 3}
\item{Kääntäjiä CPython, Jython, IronPython, PyPy}
\end{itemize}
\end{frame}

\section{Pythonin perusteita}

\begin{frame}[fragile]
\frametitle{Syntaksi}
\begin{minted}{python}
lista = [0, 1, 2, 3, 4, 5, 6, 7, 8, 9]
joukko = {0, 1, 2, 3, 4, 5, 6, 7, 8, 9}
monikko = (0, 1, 2, 3, 4, 5, 6, 7, 8, 9)

print(lista == joukko)
# False

print(set(lista) == joukko)
# True

print(lista == list(joukko) == list(monikko))
# True
\end{minted}
\end{frame}


\begin{frame}[fragile]
\frametitle{Syntaksi}
\begin{minted}[fontsize=\footnotesize]{python}
parents, babies = (1, 1)
while babies < 100:
    print 'This generation has {0} babies'.format(babies)
    parents, babies = (babies, parents + babies)

# This generation has 1 babies
# This generation has 2 babies
# This generation has 3 babies
# This generation has 5 babies
# This generation has 8 babies
# This generation has 13 babies
# This generation has 21 babies
# This generation has 34 babies
# This generation has 55 babies
# This generation has 89 babies
\end{minted}
\blfootnote{https://wiki.python.org/moin/SimplePrograms}
\end{frame}


%\begin{frame}[fragile]
%\frametitle{Duck typing}
%\end{frame}


\begin{frame}[fragile]
\frametitle{Luokat}
\begin{minted}[fontsize=\footnotesize]{python}
class MyClassA(object):
  def a(self):
    print('foo')

class MyClassB(object):
  def b(self):
    print('bar')

class MyClassC(MyClassA, MyClassB):
  """Moniperija"""
  def c(self):
    self.a()
    self.b()

olio = MyClassC()
olio.c()
# foo    # bar
\end{minted}
\end{frame}

\section{Reflektio}

\begin{frame}[fragile]
\frametitle{Introspektio}
\begin{minted}{python}
for attr in dir(olio): 
  print(attr)

# __class__
# __delattr__
# __dict__
# __doc__
# __format__
# __getattribute__
# __hash__
# ...
# a
# b
# c
\end{minted}
\end{frame}

\begin{frame}[fragile]
\frametitle{inspect-moduuli}
% \ldots
\begin{minted}[fontsize=\footnotesize]{python}
import inspect

print(len(dir(inspect)))  # 87

print(inspect.getdoc(inspect.getdoc))
# Get the documentation string for an object.
# 
# All tabs are expanded to spaces.  To clean up docstrings that are
# indented to line up with blocks of code, any whitespace than can be
# uniformly removed from the second line onwards is removed.

print(inspect.getdoc(inspect))
# Get useful information from live Python objects.   ...

print(inspect.ismodule(inspect))
# True

\end{minted}
\end{frame}

\begin{frame}[fragile]
\frametitle{inspect-moduuli}
\begin{minted}[fontsize=\footnotesize]{python}
for attr in dir(olio):
  docstr = str(inspect.getdoc(getattr(olio, attr)))
  docstr_head = docstr.splitlines()[0]
  print("olio.%s: %s" % (attr, docstr_head))

# olio.__class__: Moniperija
# olio.__delattr__: x.__delattr__('name') <==> del x.name
# olio.__dict__: dict() -> new empty dictionary
# olio.__doc__: str(object) -> string
# olio.__format__: default object formatter
# olio.__getattribute__: x.__getattribute__('name') <==> x.name
# olio.__hash__: x.__hash__() <==> hash(x)
# olio.__init__: x.__init__(...) initializes x; see help(type(x)) for signature
# ...
# olio.a: None
# olio.b: None
# olio.c: None

\end{minted}
\end{frame}

\begin{frame}[fragile]
\frametitle{inspect-moduuli}
\begin{minted}[fontsize=\footnotesize]{python}
print(inspect.getsource(olio.c))
#      def c(self):
#            self.a()
#            self.b()

print(inspect.isfunction(olio.c))
# False

print(inspect.ismethod(olio.c))
# True

print(inspect.getmro(type(olio)))
# (<class '__main__.MyClassC'>, 
#  <class '__main__.MyClassA'>, 
#  <class '__main__.MyClassB'>, 
#  <type 'object'>)

\end{minted}
\end{frame}


\begin{frame}[fragile]
\frametitle{Tavukoodin tarkastelu}
\begin{minted}{python}
import dis
dis.dis(olio.c)
# 4           0 LOAD_FAST                0 (self)
#             3 LOAD_ATTR                0 (a)
#             6 CALL_FUNCTION            0
#             9 POP_TOP             
#
# 5          10 LOAD_FAST                0 (self)
#            13 LOAD_ATTR                1 (b)
#            16 CALL_FUNCTION            0
#            19 POP_TOP             
#            20 LOAD_CONST               0 (None)
#            23 RETURN_VALUE        
\end{minted}
\end{frame}


\section{Ohjelman muokkaus ajon aikana}


\begin{frame}[fragile]
\frametitle{Ohjelman muokkaus ajon aikana}
\begin{itemize}
\item{Olioiden muokkaus sijoittamalla uusia arvoja aiempien tilalle}
\item{Luokan muokkaaminen vaikuttaa siitä luotuihin olioihin}
\item{Olemassa olevia muuttujia voidaan poistaa \verb|del|-funktiolla}
\item{Myös ulkopuolisia kirjastoja voidaan muokata vapaasti ajon aikana}
\item{Uusien luokkien luominen dynaamisesti}
\item{Koodin kääntäminen ja ajaminen}
\end{itemize}
\end{frame}


\begin{frame}[fragile]
\frametitle{Dynaamiset luokat}
\begin{minted}[fontsize=\footnotesize]{python}
def dynamic_c(self):
  self.a()
  self.b()

DynClassC = type(
  'DynClassC', 
  (MyClassA, MyClassB), 
  {'__doc__': 
   "Dynaaminen moniperinta", 
   'c': dynamic_c
  })

olio2 = DynClassC()
olio2.c()
# foo    
# bar
\end{minted}
\end{frame}

%\begin{frame}[fragile]
%\frametitle{Monkey patching}
%Esimerkiksi jonkin kirjaston toiminnallisuuden muuttaminen ajon aikana. 
% Voi olla hyödyllistä.
% TODO: Esimerkki
%\end{frame}


\begin{frame}[fragile]
\frametitle{Koodin kääntäminen ajon aikana}
\begin{minted}[fontsize=\footnotesize]{python}
code_str = 'print("Hello world!")'
code_obj = compile(code_str, '<string>', 'single')

print(list(code_obj.co_code))
# ['d', '\x00', '\x00', 'G', 'H', 'd', '\x01', '\x00', 'S']

print(dis.dis(code_obj))
#  1           0 LOAD_CONST               0 ('Hello world!')
#              3 PRINT_ITEM          
#              4 PRINT_NEWLINE       
#              5 LOAD_CONST               1 (None)
#              8 RETURN_VALUE        

exec(code_obj)
# Hello world!
\end{minted}
\end{frame}


\section{AST-puut}


\begin{frame}[fragile]
\frametitle{AST-puun käsittely}
\begin{minted}[fontsize=\footnotesize]{python}
import ast

tree = ast.parse("print('hello world')")
exec(compile(tree, filename="<ast>", mode="exec"))
# hello world

print(ast.dump(tree))
# Module( body = [Print(
#     dest = None, 
#     values = [Str(s = 'hello world')], 
#     nl = True)])

for node in ast.walk(tree):
  print(type(node))
# <class '_ast.Module'>
# <class '_ast.Print'>
# <class '_ast.Str'>

\end{minted}
\end{frame}


\begin{frame}[fragile]
\frametitle{AST-puun käsittely}
\begin{minted}[fontsize=\footnotesize]{python}
tree = ast.parse(
"""
a, b, c = ('foo', 'bar', 'baz')
print(" - ".join([a, b, c]))
""")

class StrLister(ast.NodeVisitor):
    def visit_Str(self, node):
        print(node.s)
        self.generic_visit(node)

StrLister().visit(tree)
# foo
# bar
# baz
#  - 
\end{minted}
\end{frame}


\begin{frame}[fragile]
\frametitle{AST-puun muokkaus}
\begin{itemize}
\item{Onnistuu samaan tapaan kuin \verb|NodeVisitor|-esimerkissä}
\item{Perittävänä luokkana \verb|NodeVisitor|:in sijaan \verb|NodeTransformer|}
\item{Samankaltaiset ''visit''-metodit toimivat, paluuarvona annetaan solmu (node)}
\item{\verb|ast|-moduulista löytyy apumetodeja solmujen muodostamiseen}
\end{itemize}
\end{frame}


\section{}

\begin{frame}[fragile]
  \frametitle{Yhteenveto}
\begin{itemize}
\item{Reflektio helppoa}
\item{Laajasta standardikirjastosta apua}
\item{Koodia voi kääntää ja ottaa käyttöön ajon aikana}
\item{AST-puun muokkaus}
\end{itemize}
\end{frame}


\end{document}
