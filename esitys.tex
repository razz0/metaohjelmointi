\documentclass{beamer}
\usepackage[T1]{fontenc}
\usepackage[utf8]{inputenc}
\usepackage[finnish]{babel}
\usepackage{listings}
\usepackage{minted}
\usepackage{verbatim}
\usetheme{Warsaw}

\title{Metaohjelmointi Python-kielellä}
\author{Mikko Koho}
\institute{Institute of Computer Science\\Helsingin Yliopisto}
\date{\today}
%\beamertemplatenavigationsymbolsempty

\begin{document}
\frame{\titlepage}

%\section{Sisältö}

\begin{frame}
  \frametitle{Metaohjelmointi Python-kielellä}
  \tableofcontents%[currentsection]
\end{frame}

\section{Python-kieli}

\section{Pythonin perusteita}
\begin{frame}[fragile]
\frametitle{Python}
\begin{itemize}
\item{Foo}
\item{Bar}
\end{itemize}
\end{frame}


\begin{frame}[fragile]
\frametitle{Python}
\begin{minted}{python}
print 'foobar!' # Shazbot
\end{minted}

Baz
\end{frame}

\section{Reflektio}

\begin{frame}[fragile]
\frametitle{Reflektio}
\end{frame}

\section{Ohjelman ajonaikainen muokkaus}

\begin{frame}[fragile]
\frametitle{Ohjelman ajonaikainen muokkaus}
\end{frame}

\begin{frame}[fragile]
\frametitle{Monkey patching}
Esimerkiksi jonkin kirjaston toiminnallisuuden muuttaminen ajon aikana. 
% Voi olla hyödyllistä.
% TODO: Esimerkki
\end{frame}

\section{Koodin kääntäminen ajon aikana}

\begin{frame}[fragile]
\frametitle{Koodin kääntäminen ajonaikana}
\end{frame}

\section{AST-puun muokkaus}

\begin{frame}[fragile]
\frametitle{AST-puun muokkaus}
\end{frame}


\end{document}
